\documentclass[11pt]{article}
\usepackage{amsmath,amsfonts,latexsym,graphicx}
\usepackage{fullpage,color}
%\usepackage{text}
%\usepackage{algo}
\usepackage{url,hyperref}
\usepackage{complexity}
\usepackage[linesnumbered,boxed,ruled,vlined]{algorithm2e}

\pagestyle{empty}

\setlength{\oddsidemargin}{0in}
\setlength{\topmargin}{0in}
\setlength{\textwidth}{6.5in}
\setlength{\textheight}{8.5in}

\newtheorem{fact}{Fact}
\newtheorem{lemma}{Lemma}
\newtheorem{theorem}[lemma]{Theorem}
\newtheorem{defn}[lemma]{Definition}
\newtheorem{assumption}[lemma]{Assumption}
\newtheorem{corollary}[lemma]{Corollary}
\newtheorem{prop}[lemma]{Proposition}
\newtheorem{exercise}[lemma]{Exercise}
\newtheorem{claim}[lemma]{Claim}
\newtheorem{remark}[lemma]{Remark}
\newtheorem{prob}{Problem}
\newtheorem{conjecture}{Conjecture}

\newenvironment{note}[1]{\medskip\noindent \textbf{#1:}}%
        {\medskip}

\newenvironment{proof}{\vspace{-0.05in}\noindent{\bf Proof:}}%
        {\hspace*{\fill}$\Box$\par}
\newenvironment{proofsketch}{\noindent{\bf Proof Sketch.}}%
        {\hspace*{\fill}$\Box$\par\vspace{4mm}}
\newenvironment{proofof}[1]{\smallskip\noindent{\bf Proof of #1.}}%
        {\hspace*{\fill}$\Box$\par}

\newcommand{\etal}{{\em et al.}\ }
\newcommand{\assign}{\leftarrow}

\newcommand{\opt}{\textrm{\sc OPT}}
\newcommand{\script}[1]{\mathcal{#1}}
\newcommand{\ceil}[1]{\lceil #1 \rceil}
\newcommand{\floor}[1]{\lfloor #1 \rfloor}


\begin{document}

\setlength{\fboxrule}{.5mm}\setlength{\fboxsep}{1.2mm}
\newlength{\boxlength}\setlength{\boxlength}{\textwidth}
\addtolength{\boxlength}{-4mm}
\begin{center}\framebox{\parbox{\boxlength}{\bf
IIIS 2014 Spring: ATCS - Selected Topics in Optimization \\
Lecture date: ADD DATE, 2009\\
Instructor: Jian Li   \hfill Scribe: YOUR NAME}}\end{center}
\vspace{5mm}

\section{Review}
Recall the following definition about $\epsilon$-net:
A range space $(X;R)$ is a pair consisting of an underlying universe X of objects, and a certain collection
	R of subsets (ranges) of X. 
Furthermore, we focus on those range spaces of finite  VC-dimension; namely, for any finite subset $P\subset X$, the number of distinct sets $r\cap P$,
for $r\in R$, is $O(|P|^d)$, for some constant $d$ (which is upper bounded by the VC-dimension of $(X;R)$).

Then Given a range space (X;R), a nite subset P  X, and a parameter 0 < " < 1, an "-net for P (and
R) is a subset N  P with the property that any range r 2 R with jr \ Pj  "jPj contains an element
of N. In other words, N is a hitting set for all the \heavy" ranges.

By last lecture, we have known that for any $(X;R)$, finite subset $P\subset X$ and $\epsilon$, such
	that $(X;R)$ has finite  VC-dimension $d$,
	there is a $\epsilon$-net of size $O(\frac{d}{\epsilon}\log \frac{1}{\epsilon})$. 
This can be done by a random
	sample of $P$ of that size, which could be an $\epsilon$-net with constant probability by
	double sampling tricks. 
Furthermore, the size of $\epsilon$-net for general case is tight.

\section{Construct a $\frac{1}{\epsilon}$-size $\epsilon$-net for halfspaces in $R^3$}


One of the major questions in the theory of $\epsilon$-nets is whether we could have 
	the $\epsilon$-net with a smaller size in normal (geometric) case, instead of considering the general case.
More precisely, the question is whether the factor $\log \frac{1}{\epsilon}$
	 in the upper bound on their size is really necessary.

Here we show, the answer is yes for halfspaces in $R^3$. That is, 
\begin{theorem}\cite{har2014epsilon}
	Given a set $P$ of $n$ points in $R^3$ in general position, and a parameter $0 <  \epsilon<1$, there
	exists an $\epsilon$-net for $(P;H)$ of size $O(\frac{1}{\epsilon})$, where $H$ is the family of all (closed) halfspaces
	(bounded by planes).
\end{theorem} 
It is straightforward to see that the theorem implies the same result for halfplanes in the plane. 
Further, based on projection of hyperboloid, the theorem also can imply that a similar result holds for disks in the plane. 


\subsection{Construction}
Without loss of generality, we only consider an $\epsilon$-net for lower halfspaces.

A symmetric construction will yield an $\epsilon$-net for upper halfspaces, and the union of the two nets will be
an "-net for all halfspaces. Let $H^-$ denote the set of all lower halfspaces.

\newcommand{\F}{\mathcal{F}}
Let $\beta<\frac {1} {22}$ and construct  a maximal collection $\F$ of lower halfspaces with the following properties 
\begin{enumerate}
	\item Each halfspace  $f\in \F$ contains between $\epsilon n$ and $2\epsilon n$ points.
	\item For any pair of distinct halfspaces $h,g\in \F$ , we have $ |h\cap g\cap P |\leq \beta \epsilon n$.
\end{enumerate}



Then for each halfspace $h \in F$, we construct an $\frac{\beta}{2}$-net $N_h$ for  $(h \cap P;H^-)$, of size
	$O(\frac{1}{\beta}\log \frac{1}{\beta})=O(1)$.
Therefore, we could get the union $N=\bigcup_{h\in\F} N_h$, which is the desired $\epsilon$-net.

\subsection{Proof}
To show $N$ is the desired $\epsilon$-net, 
	it is sufficient for us to prove two things: (1) $N$ is $\epsilon$-net; (2) $|\F|=O(\frac{1}{\epsilon})$. 

First, it is easy to see that $N$ is an $\epsilon$-net.
Without loss of generality, assume there exists $h\in R$ containing $\epsilon n$ points (otherwise, shift). 

	
For each h 2 F let h denote its bounding plane. By slightly perturbing these planes,
without changing any of the subsets h \ P, for h 2 F, we may assume that the planes h are in general
position. We claim that all the planes h appear on their upper envelope E. Indeed, suppose to the
contrary that there exists h 2 F such that h lies fully below the envelope. Let v be the vertex of the
envelope closest to h. Clearly, the union of the three halfspaces h1; h2; h3 2 F dening v cover h; that
is, h  h1 [ h2 [ h3. Hence, for at least one index i 2 f1; 2; 3g, we have jh \ hi \ Pj  1
3 jh \ Pj  1
3"n,
which contradicts property (b) if we choose  < 1=3.
Put t = jFj, and consider E as a planar map, which has t faces. Dene the degree deg(f) of a face
f of E, lying on some plane h, to be the number of planes g which appear on the 1-level of A(F)
directly below f (see Figure 1 for an illustration). In general, each such plane g either meets @f or
contributes a face to the 1-level which lies fully below f; the second case is impossible, though, for then
g would not appear on the upper envelope. Hence, assuming general position, deg(f) is equal to the
number of edges of f.
4

\cite{har2014epsilon}



\section{Geometric set cover}

Though the optimal set cover and hitting set
problems are NP-hard, results from $\epsilon$-nets help to give good approximation bounds for these algorithms for
simpler set systems that arise in a geometric context. 

Meggido has proved the geometric set cover problem is \NP-hard in \cite{megiddo1984complexity}. (In some cases, hardness of approximation has been shown as well).

\subsection{Unweigted version with $O(\log OPT)$ ratio}
The work is done by Clarkson and Varadarajan \cite{clarkson2007improved} by the updating weight trick, 
	which happens often in online learning field. 
	
\begin{algorithm}[h]
	\caption{Hitting set Approx for Unweighted Set Cover}
	\label{alg-set-cover-unweighted}
	Initially, let $w{x} = 1,\forall x\in X$, and let $SOL=\emptyset$\;
	\While{$SOL$ is a set cover}{
		According to the weight, pick a random subset of $R$   of size $O(\frac{\delta^*}{\epsilon} \log \frac{\delta^*}{\epsilon})$\;
		\If{$\exists p\in X$ is uncovered}
		{
			\If{$w(R_p)\leq \epsilon w(R)$}
				{
					$w(R_p)=2w(R_p)$\;
				}
		}
	}%while
	{\bf Return} $\bigoplus$\;

\end{algorithm}
\cite{har2012weighted}




BTW, there is a related note \cite{GeometricSetCover} which writes the same proof in a different way form this note. 
Readers may find more inspiring things from it.  


\section{Solve LP with fixed number of variables -- Seidel Algorithm}




\begin{algorithm}[h]
\caption{An Algorithm}
\label{algo2}
Initially, let $\mathcal{L} = 0$\;
\While{$a>b$}{
    \For{$m = 1,2,\ldots, s+1$}{
        $\Theta=\Xi$\;
    }
   $\clubsuit\diamondsuit\heartsuit\spadesuit$\;
    \If{$b<a$}{ 
        Do nothing\;
    }
}%while
{\bf Return} $\bigoplus$\;
\cite{har2012weighted}
\end{algorithm}
\cite{Seidel}
\bibliographystyle{plain}
\bibliography{a}
%% To add references, uncomment the following two lines and
%% add the relevant bibitems.

\end{document} 